\chapter*{Lista dei simboli} \addcontentsline{toc}{chapter}{Lista dei simboli}
%---------------------------------------------------------------------------------------------%
%Definizione dell'intestazione dell'introduzione
%---------------------------------------------------------------------------------------------%
\lhead[\fancyplain{}{\footnotesize{Lista dei simboli}}]{}
\rhead[]{\fancyplain{}{\footnotesize{Lista dei simboli}}}
\lfoot[\fancyplain{}{\bf \thepage}]{}
\cfoot{}
\rfoot[]{\fancyplain{}{\bf \thepage}}
%---------------------------------------------------------------------------------------------%

\section*{Simboli}

\begin{tabular}{p{1.2cm}|p{12.6cm}}
$a$                & velocit� del suono\\
$A$                & velocit� del suono adimensionale\\
$a_A$							 & livello entropico\\	
$A_A$              & livello entropico adimensionale\\
$A$                & area del connettore \\
$c_p$							 & calore specifico a pressione costante\\
$c_v$              & calore specifico a volume costante\\
$D$                & diametro del condotto\\
$E$								 & energia interna del fluido\\
$e$                & energia interna specifica del fluido\\
$f$                & coefficiente di attrito\\
$F$                & area della sezione trasversale del condotto\\
$H$                & entalpia del fluido\\
$h$                & entalpia specifica del fluido\\
$k$                & rapporto tra i calori specifici $c_p/c_v$ \\
$\dot{m}$          & portata massica\\
$p$                & pressione del fluido\\
$q$                & calore scambiato per unit� di tempo e di massa\\
$R$								 & costante equazione di stato dei gas \\
$s$                & entropia \\
$t$                & istante temporale\\
$T$								 & temperatura del gas \\
$u$                & velocit� del fluido lungo l'asse del condotto\\
$U$							   & velocit� adimensionale del fluido lungo l'asse del condotto\\	
$\mathbf{U}$       & vettore velocit� del fluido\\
$V$							   & volume\\
$x$                & ascissa curvilinea del condotto\\
$X$                & ascissa curvilinea adimensionale del condotto\\
$Z$                & tempo adimensionale\\
\end{tabular}

\newpage

\section*{Simboli}
\begin{tabular}{{p{1.2cm}|p{12.6cm}}}
$\Delta t$         & passo temporale\\
$\Delta x$         & distanza tra due nodi consecutivi del condotto\\
$\lambda$          & variabile di Riemann\\
$\beta$            & variabile di Riemann\\
$\xi$              & coefficiente di riflessione \\
$\rho$             & densit� del fluido\\
\end{tabular}



\section*{Pedici}

\begin{tabular}{{p{1.2cm}|p{12.6cm}}}
$_0$                & grandezza termodinamica totale \\
$_c$                & cella\\
$_C$								& grandezza corretta \\
$_{end}$            & grandezza riferita alla sezione terminale di un condotto\\
$_L$                & sinistra\\
$_n$ 								& componente normale\\
$_p$                & connettore\\
$_R$                & destra\\
$_{ref}$            & grandezza di riferimento\\
\end{tabular}


\section*{Apici}
\begin{tabular}{{p{1.2cm}|p{12.6cm}}}
$^+$								& onda incidente\\
$^-$								& onda riflessa\\
$^*$                & grandezza asteriscata\\
$^n$                & n-esimo istante temporale\\
\end{tabular}

%\section*{Simboli matematici}
%\begin{tabular}{{p{1.2cm}|p{12.6cm}}}
%$\partial$                & derivata parziale\\
%$\nabla$                  & divergenza \\
%\end{tabular}

\cleardoublepage
