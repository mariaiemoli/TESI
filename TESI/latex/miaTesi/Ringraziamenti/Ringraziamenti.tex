
\cleardoublepage %per cominciare sulla pagina destra

%---------------------------------------------------------------------------------------------%
%Layout dei ringraziamenti
%---------------------------------------------------------------------------------------------%
%%%%%%%%%%%%%%%%%%%%%%%
\thispagestyle{empty}%
%%%%%%%%%%%%%%%%%%%%%%%
%
\setlength{\hoffset}{-1in}
\setlength{\voffset}{-1in}

\setlength{\textheight}{211mm}
\setlength{\textwidth}{138mm}

\setlength{\evensidemargin}{32mm}
\setlength{\oddsidemargin}{41mm}
\setlength{\marginparwidth}{0mm}
\setlength{\marginparsep}{0mm}

\setlength{\topmargin}{10mm}
\setlength{\headheight}{5mm}
\setlength{\headsep}{26mm}

\setlength{\footskip}{32mm}
%---------------------------------------------------------------------------------------------%
\begin{center}
{\itshape ... Da me, da solo, solo con l'anima,\\ con la piccozza d'acciar ceruleo, \\su lento, su anelo, \\su sempre; sprezzandoti, o gelo! \\
E salgo ancora, da me facendomi \\da me la scala, tacito, assiduo; \\nel gelo che spezzo,\\ scavandomi il fine ed il mezzo. \\
\par Giovanni Pascoli}
\end{center}


\newpage %per aprire una nuova pagina e non metterne il numero di pagina
\thispagestyle{empty} %per non mettere intestazione e pi� di pagina
\cleardoublepage %per cominciare sulla pagina destra



\begin{flushright}
{\itshape A te nonna {\bfseries Anita}, \\
il tuo sorriso è il mio ricordo più caro.}

\vspace {5mm}
{\itshape Alla mia  {\bfseries famiglia}, \\
per l'immensa bellezza della sua semplicità.}

\end{flushright}

\vspace{6mm}

\noindent
\\\\
\\\\
{\itshape 
Ringraziamenti} 
\begin{flushright}
{\itshape \bfseries Stefania}
\end{flushright}


\newpage %per aprire una nuova pagina e non metterne il numero di pagina
\thispagestyle{empty} %per non mettere intestazione e pi� di pagina
\cleardoublepage %per cominciare sulla pagina destra




