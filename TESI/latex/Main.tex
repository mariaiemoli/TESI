\documentclass[a4paper,11pt,openright,oneside]{book}
\frenchspacing

\usepackage[utf8]{inputenc}
\usepackage[italian]{babel}
\usepackage[dvips]{graphicx}
\usepackage{url}
\usepackage{rotating}
\usepackage{verbatim}
\usepackage{longtable}
\usepackage[boxed]{algorithm2e}
\usepackage{amsmath}

\usepackage{amssymb} %simboli matematici
\usepackage{latexsym} %?
\usepackage{mathrsfs} %?
\usepackage{amsfonts}

\usepackage{hyperref}

\usepackage{frontespizio}
\usepackage{listings} %Per inserire codice
\usepackage[usenames]{color}%Per permettere la colorazione dei caratteri 
\usepackage{fancyhdr} 


\usepackage{listings} %Per inserire codice
\usepackage[usenames]{color}%Per permettere la colorazione dei caratteri 

\usepackage{titlesec,blindtext}
\usepackage[Lenny]{fncychap}

\usepackage{eso-pic}
\newcommand\BackgroundPicture{
   \put(0,0){
     \parbox[b][\paperheight]{\paperwidth}{
       \vfill
       \centering
       \includegraphics[width=\textwidth]{img/fratture.eps}
       \vfill
     }}}



%\usepackage[Sonny]{fncychap}
%\usepackage[toc]{glossaries}
%\makeglossaries
%\usepackage{lscape}

\ChNameVar{\fontsize{14}{16}\usefont{T1}{phv}{m}{n}\selectfont}
\ChNumVar{\fontsize{60}{62}\usefont{T1}{ptm}{m}{n}\selectfont}

 
\begin{document}

\begin{frontespizio}

\Istituzione{POLITECNICO DI MILANO}
\Logo{logo_polimi}
\Facolta{Ingegneria Industriale e dell'informazione}
\Corso[Laurea]{di Laurea Magistrale in INGEGNERIA MATEMATICA}
\Annoaccademico{2013-2014}
\Sottotitolo{Imposizione delle condizioni d'interfaccia per intersezioni di tipo Biforcazione}
\Titolo{\color[rgb]{.6,0,0} Soluzione del problema di Saturazione in un network di fratture}
\Candidato[]{Maria Iemoli matr. 800499}
\Titoletto{Tesi di Laurea Magistrale}
\Relatore{ Luca Formaggia}
\Relatore{ Anna Scotti}
\Rientro{1cm}
\Margini{1.5cm}{2cm}{1.5cm}{3cm}
\end{frontespizio} 


\hypersetup{
    %colorlinks,
    citecolor=black,
    filecolor=black,
    linkcolor=black,
    urlcolor=black
}

\emptypage
\chapter*{Abstract}

L'obiettivo di questa tesi...
\newpage 
\chapter*{Abstract}

The aim of this work...
\newpage 

\include{introduzione}

\pagestyle{plain}
\pagenumbering{Roman}
\include{prefazione}

\pagenumbering{arabic}
\tableofcontents

\include{cap1}
\include{cap2}
\include{cap3}
\include{cap4}
\include{cap5}

%\include{bibliografia.bib}

\bibliographystyle{plain}
\bibliography{bibliografia}

\include{appendice}

\end{document}




\begin{tikzpicture}[remember picture,overlay]
\node [opacity=0.2,scale=0.2] at (current page.center) {\includegraphics{img/fratture.eps}};
\end{tikzpicture}
